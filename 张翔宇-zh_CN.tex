% !TEX TS-program = xelatex
% !TEX encoding = UTF-8 Unicode
% !Mode:: "TeX:UTF-8"

\documentclass{resume}
\usepackage{zh_CN-Adobefonts_external} % Simplified Chinese Support using external fonts (./fonts/zh_CN-Adobe/)
%\usepackage{zh_CN-Adobefonts_internal} % Simplified Chinese Support using system fonts
\usepackage{linespacing_fix} % disable extra space before next section
\usepackage{cite}
\usepackage{hyperref}
\hypersetup{
    colorlinks=true,
    linkcolor=blue,
    filecolor=magenta,      
    urlcolor=cyan,
}
\begin{document}
\pagenumbering{gobble} % suppress displaying page number

\name {张翔宇}

\basicInfo{
  \email{neuzxy@126.com} \textperiodcentered\ 
  \phone{(+86) 188-1173-1628} \textperiodcentered\ 
  \href{https://github.com/neuzxy}{GitHub}}
 
\section{\faGraduationCap\  教育背景}
\datedsubsection{\textbf{北京大学}}{2014.09 -- 2017.07}
\textit{工学硕士}\ 保送软件工程一级学科学术型硕士
\datedsubsection{\textbf{东北大学}}{2010.09 -- 2014.07}
\textit{工学学士}\ 软件工程

\section{\faUsers\ 项目经历}
% (工作9个月内部升级15级)
\datedsubsection{\textbf{华为杭州研究所} CloudBU \href{https://www.huaweicloud.com/product/dls.html} {深度学习组}(入职九个月升级)}{2017 年 7 月 – 至今}
\role{Java, Scala, Akka HTTP, Play Framework, Docker, Kubernetes, Tensorflow etc.}{{深度学习云服务后端开发}}

\begin{onehalfspacing}
\textbf{DLS Server}  深度学习服务端组件
\begin{itemize}%[parsep=0.5ex]
  \item 深度学习算法模型库模块的设计与开发,完成用户视角和后台管理视角 RESTful API 开发。
  \item 对接告警服务、日志服务和调用链监控服务,定位线上问题。
  \item 技术栈: Scala, Play Framework, Slick, PostgreSQL, AWS S3, SBT
\end{itemize}
\end{onehalfspacing}

\begin{onehalfspacing}
\textbf{DLS Agent} 深度学习服务跨域消息转发组件
\begin{itemize}%[parsep=0.5ex]
  \item 组件功能开发。设计与实现了对接 DLKS Server(深度学习服务作业调度组件),DLS File Server(深度学习文件管理组件) 和 CES(公有云监控服务) 等组件的 27 个 RESTful API 开发。
  \item 组件安全增强。完成 Server-Side 和 Client-Side 对 https 的支持,MySQL Client 对 SSL 的支持。
  \item 技术栈: Java, Akka HTTP, gRPC, MySQL, Maven
\end{itemize}
\end{onehalfspacing}

\begin{onehalfspacing}
\textbf{DLS Proxy\&DLS Gateway}  深度学习服务跨域WebSocket/HTTP代理组件
\begin{itemize}%[parsep=0.5ex]
  \item DLS Proxy组件。实现对Jupyter Notebook和Tensorboard请求的消息转发。
  \item DLS Gateway组件。实现IAM Token校验机制以及对Jupyter Notebook和Tensorboard的消息转发。
  \item 技术栈: Java, Spring Boot, Tomcat, Websocket, Netflix Zuul
\end{itemize}
\end{onehalfspacing}

\begin{onehalfspacing}
\textbf{其它相关工作}  
\begin{itemize}%[parsep=0.5ex]
  \item 开源贡献。TensorFlow 对 AWS S3 文件系统的支持(重构/BUG/UT),贡献 \href{https://github.com/tensorflow/tensorflow/pull/11089#issuecomment-320258536}{3 个patches}(C++)。
  \item 公共工具。Java 和 Scala 组件\href{https://github.com/neuzxy/AESCryptTools}{加密工具} ,Kubernetes对各组件的健康检查(HTTP/gRPC Client)。 
  \item 华为云 MXNet 引擎推理服务\href{https://github.com/huawei-clouds/dls-mxserving-client/tree/master/java}{Java 客户端}。
\end{itemize}
\end{onehalfspacing}

\datedsubsection{\textbf{MOOC 用户行为分析}}{2016 年 10 月 – 2017 年 4 月}
\role{Python, Pandas, Scikit-learn, XGBoost} {毕业论文}
\begin{itemize}[parsep=0.5ex]
  \item 用户流失预测\&基于点击流的学习行为分析。基于用户点击内容和停留时间对点击流建模,挖掘和分析用户的行为模式。发表 \href{https://link.springer.com/chapter/10.1007/978-981-10-7398-4_41}{一作EI 论文一篇}. 
\end{itemize}

\datedsubsection{\textbf{360 新闻部实习}}{2016 年 3 月 – 2016 年 6 月}
\role{Python, Scala, Spark, Redis, MySQL }{新闻推荐算法工程师}
\begin{itemize}[parsep=0.5ex]
  \item 离线日志分析。新闻 CTR多维统计分析,线上待推荐新闻类别和质量多维度分析统计。
  \item 用户画像工作。批量生成用户画像,分析Case。线上用户画像增量更新,优化后时间缩短 80\%。
\end{itemize}

\section{\faCogs\ IT 技能}
% increase linespacing [parsep=0.5ex]
\begin{itemize}[parsep=0.5ex]
  \item 编程语言: Java > C++ = Scala = Python 英语六级 561
  \item 开发: 熟悉 Linux 平台, 了解高可用服务系统架构和分布式系统原理, 了解常见的机器学习算法。
\end{itemize}

\end{document}

% Reference Test
%\datedsubsection{\textbf{Paper Title\cite{zaharia2012resilient}}}{May. 2015}
%An xxx optimized for xxx\cite{verma2015large}
%\begin{itemize}
%  \item main contribution
%\end{itemize}


%\section{\faHeartO\ 获奖情况}
%\datedline{\textit{第一名}, xxx 比赛}{2013 年6 月}
%\datedline{其他奖项}{2015}
%
%\section{\faInfo\ 其他}
%% increase linespacing [parsep=0.5ex]
%\begin{itemize}[parsep=0.5ex]
%  \item 技术博客: http://blog.yours.me
%  \item GitHub: https://github.com/username
%  \item 语言: 英语 - 熟练(TOEFL xxx)
%\end{itemize}

%% Reference
%\newpage
%\bibliographystyle{IEEETran}
%\bibliography{mycite}

%•Java>C++=Scala>Python(实际项目经验)•开发:计算机基础扎实,代码能力强,有分布式后台开发和算法研发经验。熟悉常见的统计机器学习算法,了解分布式系统原理。经历较杂,想在一个领域做深求职意向:1、分布式平台建设(深度学习、机器学习、大数据平台)。2、基础软件研发(文件存储、数据库、RPC框架)
%
%2017 年 7 月 – 至今 深度学习云服务后端开发 
%编程语言:Java, Scala, C++, Python
%软件和框架:Akka HTTP, Play Framework, Docker, Kubernetes, Tensorflow, gRPC, MySQL, Maven, SBT, Git etc.
%
%1. DLS Agent 深度学习服务消息转发组件(跨域接口层组件)  组件负责人
%  • 组件功能开发。完成组件技术选型,实现对接 DLKS Server(深度学习服务作业调度组件),DLKS Monitor(深度学习作业监控组件)、DLS File Server(深度学习文件管理组件) 和 CES(公有云监控服务) 的 27 个 RESTful API 开发。
%• 组件安全增强。完成 Server-Side 和 Client-Side 对 https 的支持,MySQL Client 对 SSL 的支持。
%• 技术栈:Java, Akka HTTP, gRPC, MySQL, Maven, Docker
%
%2. DLS Server 深度学习服务端组件 (算法模型库模块)模块负责人
%• 深度学习算法模型库模块的设计与开发,完成模型库的用户视角和后台管理视角 RESTful API 开发。 
%• 对接调用链服务,定位线上问题。
%• 技术栈:Scala, Play Framework, Slick, PostgreSQL, AWS S3, SBT
%
%3. 其他相关工作
%• 开源贡献。TensorFlow 对 AWS S3 文件系统的支持,贡献 3 个patches(C++11)。
%• 项目组公共服务。Java 和 Scala 组件加密工具(aes-128-cbc) 
%• Kubernetes 对深度学习组件的健康检查(Java, Python gRPC client)。
%• 华为云 MXNet 引擎图片分类服务Java 客户端(HTTP Client)。
%• 对DLKS Server(深度学习服务作业调度组件 C++11)、DLS File Server(深度学习文件管理组件 Java8)也有部分特性代码贡献。

%• 东北大学优秀毕业生 (Top 4%) 
%• 国家奖学金 (Top 2%)
%• 东北大学二等奖学金 (Top 10%) 
%• 东北大学一等奖学金 (Top 3%)