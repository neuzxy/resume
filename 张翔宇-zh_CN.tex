% !TEX TS-program = xelatex
% !TEX encoding = UTF-8 Unicode
% !Mode:: "TeX:UTF-8"

\documentclass{resume}
\usepackage{zh_CN-Adobefonts_external} % Simplified Chinese Support using external fonts (./fonts/zh_CN-Adobe/)
%\usepackage{zh_CN-Adobefonts_internal} % Simplified Chinese Support using system fonts
\usepackage{linespacing_fix} % disable extra space before next section
\usepackage{cite}
\usepackage{hyperref}
\hypersetup{
    colorlinks=true,
    linkcolor=blue,
    filecolor=magenta,      
    urlcolor=cyan,
}
\begin{document}
\pagenumbering{gobble} % suppress displaying page number

\name {张翔宇}

\basicInfo{
  \email{neuzxy@126.com} \textperiodcentered\ 
  \phone{(+86) 188-1173-1628} \textperiodcentered\ 
  \href{https://github.com/neuzxy}{GitHub}
  }
 
\section{\faGraduationCap\  教育背景}
\datedsubsection{\textbf{北京大学}}{2014.09 -- 2017.07}
\textit{工学硕士}\ 保送软件工程一级学科学术型硕士
\datedsubsection{\textbf{东北大学}}{2010.09 -- 2014.07}
\textit{工学学士}\ 软件工程

\section{\faCogs\ IT 技能}
% increase linespacing [parsep=0.5ex]
\begin{itemize}[parsep=0.5ex]
  % \item 编程语言C++ = Python = Java > Scala, 英语六级 561。
  %\item 熟悉 Linux 平台,熟悉高可用服务系统架构和分布式系统原理。
  \item 曾在华为(华为云深度学习平台、华为诺亚方舟实验室)和百度凤巢CTR模型组工作,4年推荐广告模型和大规模深度学习平台研发经验,熟悉TensorFlow和Paddle等深度学习引擎,优化多个Paddle Op,为TensorFlow贡献过3 个\href{https://github.com/tensorflow/tensorflow/pull/11089#issuecomment-320258536}{patches}。
  \item 寻求对标阿里P7相关岗位(当前薪酬已在区间内)。
\end{itemize}

\section{\faUsers\ 项目经历}
\datedsubsection{\textbf{百度} {凤巢CTR模型中台}}{2020.6 –  至今}
\role{C++, Python, CUDA, NCCL, Paddle etc.}{{研发工程师}}
\begin{onehalfspacing}
%\textbf{训练框架开发} 
\textbf{负责AIBox/PaddleBox训练框架开发与多个业务的支撑}。PaddleBox是基于SSD/MEM/GPU的异构参数服务器训练框架,通过软硬件联合优化来支撑万亿参数DNN模型的Online Learning,目前支撑了百度商业化搜索广告、信息流原生广告等数十个场景的CTR模型训练。个人获得\href{https://baidu.com}{\textbf{百度移动生态部MEG最佳新人奖}}、联合团队获得\textbf{百度技术委员会技术创新奖},作为作者之一(团队内部排第三)将项目成果 <<Communication-Efficient TeraByte-Scale Model Training Framework for Online Advertising>> \textbf{投稿VLDB}。
\begin{itemize}%[parsep=0.5ex
\item \textbf{训练框架开发}

\begin{itemize}
\item \textbf{性能优化} 。通过Op Fusion等技术优化多个Paddle GPU Op (e.g. \href{https://github.com/neuzxy/Paddle/commit/5e6f8bf1f38622571d868e35bb99ccb924856aee}{fused embedding、continue value} 使得DNN训练整体提升15\%),引入类似Horovod的Tensor Fusion技术优化梯度通信(性能提升5\%),DNN训练性能累计提升超过20\%。
\item \textbf{Embedding量化}。通过Embedding量化来优化存储和计算性能,与搜索广告主CTR模型合作把Embedding从8维扩到16维提高模型表征能力而参数存储空间不变(FP32->INT16),CPM提升超过1\%。\href{https://github.com/neuzxy/Paddle/commit/1c5c0591ec3402c31eaa32a31c62e9fc518a11f2}{Paddle代码}
\item \textbf{其他特性开发}。参数服务器统一显存管理;Sparse参数Adam GPU优化器,支撑TDM召回解决方案;Dense模型转换等相关工具开发。
\end{itemize}

\item \textbf{分布式SSD参数服务器架构升级}
\begin{itemize}
\item \textbf{项目背景}。PaddleBox架构为单进程架构,GPU Worker和异构参数服务器位于同一进程空间内,其中SSD KV引擎存储全量模型,SSD参数查找是很多场景模型训练的瓶颈。%训练流程为样本读取、SSD参数查找、DNN训练三级流水线并行,
\item \textbf{方案设计与开发}。设计SSD参数服务器独立架构,支持\textbf{GPU Worker和SSD参数服务器M:N}。架构升级涉及SSD参数查找和更新、模型加载、全量模型保存和增量模型保存等功能开发,通过多种方式优化\textbf{SSD-PS PULL性能达到超线性加速比}。% 把SSD参数服务器做为可横行扩展的独立服务来提升SSD参数查找效率
% 通过高效RPC通信、设计mempool、CPU binding、PULL/PUSH无锁消息队列和充分利用NUMA架构等技术进一步优化训练性能。
\item \textbf{项目收益}。分布式SSD参数服务器PULL的性能可以达 \textbf{超线性加速比},通过调配GPU Worker和SSD参数服务器数量,\textbf{灵活应对算力瓶颈和参数拉取瓶颈}。架构升级后,搜索广告主模型内存占用从单进程的1.2T降到700G(机器内存1.5TB),可通过\textbf{模型混部}进一步提高资源利用率(进行中)。%(小于机器内存一半)
%架构解耦改造涉及模型下载、SSD参数查找、SSD参数更新、全量模型保存和增量模型保存等功能,通过Trainer本地参数缓存、PULL/PUSH充分并行、绑核和无锁队列等优化,1个单机8卡的GPU Trainer、2个SSD-PS模型相比单进程架构训练性能提升80\%。
\end{itemize}

\item \textbf{分布式GPU参数服务器}
\begin{itemize}
\item \textbf{项目背景}。分布式训练需要通过RPC预取分布在所有节点的Embedding参数,训练时直接从本机GPU Hashtable读取,不同机器GPU参数存储有冗余导致显存利用率不高。
\item \textbf{方案设计与开发}。设计GPU多对多架构通信架构,通过参数存储无冗余的分布式GPU Hashtable方案来减少GPU显存占用,通过多种方式优化Embedding参数跨机跨卡访存性能。
% 单GPU卡需要的Embedding参数分布在所有GPU上,通过对输入数据sharding,节点间利用GDR AlltoAll、节点内利用GPUDirect P2P通信来提升效率,通过相同参数合并减少通信量进一步提升性能。
\item \textbf{项目收益}。HBM PS显存占用显著减少,约为原来的1/N(N为节点数),可以支撑更大DNN模型。
\end{itemize}

\end{itemize}

\textbf{业务推广} 
\begin{itemize}%[parsep=0.5ex]
 \item \textbf{商业化垂类业务支撑}。支撑贴吧等业务从基于MPI的CPU参数服务器框架迁移到PaddleBox。
 \item \textbf{商业化反作弊业务支撑}。支撑多个业务从XGBoost 树模型转到PaddleBox DNN模型,开发以连续特征为主、离散特征为辅的DNN模型。
\end{itemize}

\end{onehalfspacing}

\datedsubsection{\textbf{华为 - 共两次晋升} \href{http://www.noahlab.com.hk}{诺亚方舟实验室}}{2018年9月 –  2020年6月}
\role{Tensorflow, Hive, Spark etc.}{{高级推荐算法工程师}}

\begin{onehalfspacing}
\textbf{华为手机应用市场(亿级月活产品)App付费推广业务} 。参与Offline、Nearline和Online三层推荐架构开发,历经FTRL、FFM和DNN类模型的迭代优化,从数据、特征、模型和系统层面端到端优化提升业务指标。  
\begin{itemize}%[parsep=0.5ex]
 % \item 数据层面。客户端精准曝光数据建模,反作弊去除恶意刷量异常样本。%eCPM提升2\%。
  \item \textbf{数据和特征层面}。采样策略探索,线上线下特征一致化;挖掘Item、User、Context单特征和组合特征,debias建模。
  %新的有效特征,添加App评论数、评分、最近下载量/下载率、搜索词和关联度等单特征和组合特征,用于FTRL和Deep \& Wide模型,线上
  %\item \textbf{模型层面}。优化基于OpenMP的FTRL算法库,支持十亿特征、百亿样本小时级训练,优化DeepFM和DCN多卡训练性能。
  \item \textbf{模型层面}。历经FTRL、FFM和Wide\&Deep类模型迭代优化,探索CTR、CVR 联合建模,多榜单联合建模,eCPM累计提升超过15\%。
  %\item \textbf{策略层面}。设计多路召回、动态候选集、CTR后处理规则、多样性等策略。
  \item \textbf{系统层面}。
    \begin{itemize}
    \item \textbf{训练优化}。基于OpenMP的FTRL训练优化,支撑单机百亿样本小时级训练;深度学习训练优化,构建Spark生成TFRecord Pipeline,优化深度学习数据读取模块,落地多卡训练,单机四卡训练线性加速比0.7。数据IO优化方案现用于\href{https://github.com/mindspore-ai/mindspore/blob/master/model_zoo/official/recommend/wide_and_deep/src/datasets.py}{华为自研深度学习框架mindspore ModelZoo}。
    \item \textbf{线上服务优化}。Java线上服务开发,利用TF Batching机制推理性能优化,推理性能提升40\%。
    \item \textbf{分布式推理探索}。分布式TensorFlow-Serving推理原型开发验证(多模型副本),高并发情况下1000+候选集预测线性加速比0.77。\href{https://github.com/neuzxy/Distributed-TF-Serving.git}{Demo}
  \end{itemize}
  % \textit{离线训练优化}:FTRL特征工程实验效率提升数倍,落地深度学习模型多卡训练,四卡线性加速比0.6;\textit{线上服务优化}:大规模候选集场景下的在线推理服务latency优化,其中低秩模型(FTRL、FFM)采用多线程加速,线上服务latency减少50\%。深度学习模型基于TensorFlow-Serving的Batch Predict机制,较通过JNI调用libtensorflow.so性能提升40\%。
%  \item \textbf{分布式系统原型探索}。
%      \begin{itemize}
%    \item \textbf{分布式推理探索}。分布式TensorFlow-Serving推理原型开发验证(多模型副本),高并发情况下1000+候选集预测线性加速比0.77。
%    \item \textbf{分布式训练探索}。基于PS-Lite参数服务器的分布式FTRL调研与测试,TensorFlow PS、Horovod多机多卡探索和性能优化。
%    \end{itemize}
  % \textit{推理优化}:分布式TensorFlow-Serving推理原型开发验证,高并发情况下1500候选集预测线性加速比0.77; \textit{训练优化}:基于ps-lite参数服务器的分布式FTRL和FFM算法调研与测试,TensorFlow多机多卡探索和性能优化。
\end{itemize}
\end{onehalfspacing}

\begin{onehalfspacing}
\textbf{原生广告业务}   
\begin{itemize}%[parsep=0.5ex]
  \item \textbf{智能助手原生广告}。负责智能助手业务(信息流广告)数据、特征和模型的端到端优化,引入文本和图片多模态特征,上线初版FTRL模型和DCN模型,CTR点击率提升累计超过20\%,eCPM提升超过15\%。
%  \item \textbf{华为视频App推广}。负责华为视频App推广的迭代优化,缓解样本极不均衡问题,上线FTRL和FFM模型,eCPM累计提升超过25\%。
\end{itemize}
\end{onehalfspacing}

\begin{onehalfspacing}
\textbf{推荐多样性算法研究与落地}   
\begin{itemize}%[parsep=0.5ex]
  \item \textbf{算法研究}。提出一种个性化的DPP算法(Determinantal Point Processes,行列式点过程),在不影响精度的情况下提升推荐多样性。
 研究成果\href{https://arxiv.org/abs/2004.06390}{Personalized Re-ranking for Improving Diversity in Live Recommender Systems}中稿KDD Workshop(DLP-KDD 2020),第二作者。
  \item \textbf{应用市场猜你喜欢场景业务落地}。负责整体方案设计,线上服务代码开发与测试,其中原始DPP下载转化率较基线提升5.5\%,提出的个性化DPP下载转化率较基线提升6.5\%。
\end{itemize}
\end{onehalfspacing}

%\begin{onehalfspacing}
%\textbf{其它相关工作}  
%\begin{itemize}%[parsep=0.5ex]
%  \item \textbf{高校合作项目}。
% % \item \textbf{浏览器信息流召回算法}。参与Item-CF和ALS召回算法实验和落地。
%\end{itemize}
%\end{onehalfspacing}

% (工作9个月内部升级15级) ()
\datedsubsection{\textbf{华为} \href{https://www.huaweicloud.com/product/modelarts.html} {ModelArts深度学习云服务}(从零到一,九个月晋升)}{2017年7月 –  2018年9月}
\role{Java, Docker, Kubernetes, Tensorflow etc.}{{深度学习云平台后端开发}}

\begin{onehalfspacing}
\textbf{服务端组件}    
\begin{itemize}%[parsep=0.5ex]
  \item \textbf{组件功能开发}。负责训练作业和预置算法模型库等模块的设计与开发,支持TensorFlow、MXNet、PyTorch和Horovod等深度学习作业的生命周期管理,支持多种深度学习引擎、多种版本模型库。%,支持模型的多引擎多版本
  \item \textbf{服务可靠性}。对接公有云统一告警服务、调用链监控服务、云审计和日志收集服务,无状态多实例容器化部署。
  % \item \textbf{技术栈}: Scala, PostgreSQL, AWS S3
\end{itemize}

\begin{onehalfspacing}
\textbf{作业调度组件}  
\begin{itemize}%[parsep=0.5ex]
  \item \textbf{组件功能开发}。参与基于Kubernetes的深度学习作业调度和监控组件设计与开发,实现对多引擎、多类型深度学习作业的全生命周期管理,负责部分gPRC后端接口的实现,对外提供gPRC服务。
  % \item \textbf{技术栈}: C++, gRPC, Docker, Kubernetes, MySQL
\end{itemize}

\begin{onehalfspacing}
\textbf{消息(HTTP/gRPC)转发组件}  
\begin{itemize}%[parsep=0.5ex]
  \item \textbf{组件功能开发}。负责技术选型,对接作业调度组件和文件管理组件gRPC服务,向外提供RESTful API。
  %\item 组件安全增强。完成 Server-Side 和 Client-Side 对 https 的支持,MySQL Client 对 SSL 的支持。
  \item \textbf{服务可靠性}。Server-Side、Client-Side对SSL 的支持,数据库定时心跳检测,作业状态上报失败重试。
  % \item \textbf{技术栈}: Java, Akka HTTP, gRPC, MySQL
\end{itemize}
\end{onehalfspacing}

\end{onehalfspacing}

\end{onehalfspacing}


%\begin{onehalfspacing}
%\textbf{跨域WebSocket/HTTP代理组件}  
%\begin{itemize}%[parsep=0.5ex]
%  \item 组件功能开发。设计与实现Jupyter Notebook和Tensorboard请求的多Region转发(公有云公共前端组件console不支持Websocket协议),通过Token校验机制保障访问安全性,实现Websocket连接限流。
%  \item 技术栈: Java, Spring Boot,  Websocket, Netflix Zuul
%\end{itemize}
%\end{onehalfspacing}

\begin{onehalfspacing}
\textbf{其它相关工作}  
\begin{itemize}%[parsep=0.5ex]
  \item \textbf{开源贡献}。TensorFlow 对 AWS S3 文件系统的支持(重构/BUG FIX/UT),贡献 \href{https://github.com/tensorflow/tensorflow/pull/11089#issuecomment-320258536}{3 个patches}(C++)。
  %\item \textbf{公共工具}。Java 和 Scala 服务组件\href{https://github.com/neuzxy/AESCryptTools}{加密工具} 。 
  % \item 华为云 MXNet 引擎推理服务\href{https://github.com/huawei-clouds/dls-mxserving-client/tree/master/java}{Java 客户端}。
  \item MXNet NCCL/RDMA高性能技术调研和性能测试。
\end{itemize}
\end{onehalfspacing}

\end{document}
%\datedsubsection{\textbf{MOOC 用户行为分析}}{2016 年 10 月 – 2017 年 4 月}
%\role{Python, Pandas, Scikit-learn, XGBoost} {毕业论文}
%\begin{itemize}[parsep=0.5ex]
%\item 用户流失预测\&基于点击流的学习行为分析。基于用户点击内容和停留时间对点击流建模,挖掘和分析用户的行为模式。发表 \href{https://link.springer.com/chapter/10.1007/978-981-10-7398-4_41}{一作EI 论文一篇}. 
%\end{itemize}
%
%\datedsubsection{\textbf{360 新闻推荐}}{2016 年 3 月 – 2016 年 6 月}
%\role{Python, Scala, Spark, Redis, MySQL }{新闻推荐算法工程师}
%\begin{itemize}[parsep=0.5ex]
%\item \textbf{离线数据Pipeline}。构建新闻特征和用户行为信息抽取Pipeline,Spark性能优化,优化后时间缩短 80\%。
%\item \textbf{召回和排序}。构建基于类目、关键词、地域和热度多路召回策略,LR排序模型的特征优化。
%\end{itemize}
%

%\end{document}

% Reference Test
%\datedsubsection{\textbf{Paper Title\cite{zaharia2012resilient}}}{May. 2015}
%An xxx optimized for xxx\cite{verma2015large}
%\begin{itemize}
%  \item main contribution
%\end{itemize}


%\section{\faHeartO\ 获奖情况}
%\datedline{\textit{第一名}, xxx 比赛}{2013 年6 月}
%\datedline{其他奖项}{2015}
%
%\section{\faInfo\ 其他}
%% increase linespacing [parsep=0.5ex]
%\begin{itemize}[parsep=0.5ex]
%  \item 技术博客: http://blog.yours.me
%  \item GitHub: https://github.com/username
%  \item 语言: 英语 - 熟练(TOEFL xxx)
%\end{itemize}

%% Reference
%\newpage
%\bibliographystyle{IEEETran}
%\bibliography{mycite}

%•Java>C++=Scala>Python(实际项目经验)•开发:计算机基础扎实,代码能力强,有分布式后台开发和算法研发经验。熟悉常见的统计机器学习算法,了解分布式系统原理。经历较杂,想在一个领域做深求职意向:1、分布式平台建设(深度学习、机器学习、大数据平台)。2、基础软件研发(文件存储、数据库、RPC框架)
%
%2017 年 7 月 – 至今 深度学习云服务后端开发 
%编程语言:Java, Scala, C++, Python
%软件和框架:Akka HTTP, Play Framework, Docker, Kubernetes, Tensorflow, gRPC, MySQL, Maven, SBT, Git etc.
%
%1. DLS Agent 深度学习服务消息转发组件(跨域接口层组件)  组件负责人
%  • 组件功能开发。完成组件技术选型,实现对接 DLKS Server(深度学习服务作业调度组件),DLKS Monitor(深度学习作业监控组件)、DLS File Server(深度学习文件管理组件) 和 CES(公有云监控服务) 的 27 个 RESTful API 开发。
%• 组件安全增强。完成 Server-Side 和 Client-Side 对 https 的支持,MySQL Client 对 SSL 的支持。
%• 技术栈:Java, Akka HTTP, gRPC, MySQL, Maven, Docker
%
%2. DLS Server 深度学习服务端组件 (算法模型库模块)模块负责人
%• 深度学习算法模型库模块的设计与开发,完成模型库的用户视角和后台管理视角 RESTful API 开发。 
%• 对接调用链服务,定位线上问题。
%• 技术栈:Scala, Play Framework, Slick, PostgreSQL, AWS S3, SBT
%
%3. 其他相关工作
%• 开源贡献。TensorFlow 对 AWS S3 文件系统的支持,贡献 3 个patches(C++11)。
%• 项目组公共服务。Java 和 Scala 组件加密工具(aes-128-cbc) 
%• Kubernetes 对深度学习组件的健康检查(Java, Python gRPC client)。
%• 华为云 MXNet 引擎图片分类服务Java 客户端(HTTP Client)。
%• 对DLKS Server(深度学习服务作业调度组件 C++11)、DLS File Server(深度学习文件管理组件 Java8)也有部分特性代码贡献。

%• 东北大学优秀毕业生 (Top 4%) 
%• 国家奖学金 (Top 2%)
%• 东北大学二等奖学金 (Top 10%) 
%• 东北大学一等奖学金 (Top 3%)