% !TEX TS-program = xelatex
% !TEX encoding = UTF-8 Unicode
% !Mode:: "TeX:UTF-8"

\documentclass{resume}
\usepackage{zh_CN-Adobefonts_external} % Simplified Chinese Support using external fonts (./fonts/zh_CN-Adobe/)
%\usepackage{zh_CN-Adobefonts_internal} % Simplified Chinese Support using system fonts
\usepackage{linespacing_fix} % disable extra space before next section
\usepackage{cite}
\usepackage{hyperref}
\hypersetup{
    colorlinks=true,
    linkcolor=blue,
    filecolor=magenta,      
    urlcolor=cyan,
}
\begin{document}
\pagenumbering{gobble} % suppress displaying page number

\name{张翔宇}

\basicInfo{
  \email{neuzxy@126.com} \textperiodcentered\ 
  \phone{(+86) 188-1173-1628} \textperiodcentered\ 
  \linkedin[张翔宇]{https://www.linkedin.com/in/pkuzxy/}}
 
\section{\faGraduationCap\  教育背景}
\datedsubsection{\textbf{北京大学}}{2014.09 -- 2017.07}
\textit{工学硕士}\ 保送软件工程一级学科学术型硕士
\datedsubsection{\textbf{东北大学}}{2010.09 -- 2014.07}
\textit{工学学士}\ 软件工程,Top 3\%

\section{\faUsers\ 项目经历}

\datedsubsection{\textbf{华为杭州研究所} CloudBU 深度学习组}{2017 年 7 月 – 至今}
\role{Java, Scala, Akka HTTP, Play Framework, Docker, Kubernetes, Tensorflow etc.}{{深度学习云服务后端开发}}

\begin{onehalfspacing}
\textbf{DLS Agent} 深度学习服务 Pod 域消息转发组件
\begin{itemize}%[parsep=0.5ex]
  \item 组件功能开发。设计与实现了对接 DLKS Server(深度学习服务作业调度组件),DLS File Server(深度学习文件管理组件) 和 CES(公有云监控服务) 等组件的 27 个 RESTful API 开发。
  \item 组件安全增强。完成 Server-Side 和 Client-Side 对 https 的支持,MySQL Client 对 SSL 的支持。
  \item 技术栈: Java, Akka HTTP, gRPC, MySQL, Maven
\end{itemize}
\end{onehalfspacing}

\begin{onehalfspacing}
\textbf{DLS Server}  深度学习服务端组件
\begin{itemize}%[parsep=0.5ex]
  \item 深度学习算法模型库模块的设计与开发,完成用户视角和后台管理视角 RESTful API 开发。
  \item 对接华为云调用链监控服务(正在进行),定位线上问题。
  \item 技术栈: Scala, Play Framework, Slick, PostgreSQL, AWS S3, SBT
\end{itemize}
\end{onehalfspacing}

\begin{onehalfspacing}
\textbf{其它相关工作}  
\begin{itemize}%[parsep=0.5ex]
  \item 开源贡献。TensorFlow 对 AWS S3 文件系统的支持,贡献 \href{https://github.com/tensorflow/tensorflow/pull/11089#issuecomment-320258536}{3 个patches}。
  \item 项目组公共服务。Java 和 Scala 组件\href{https://github.com/neuzxy/AESCryptTools}{加密工具} ,Kubernetes 对深度学习组件的健康检查。 
  \item 华为云 MXNet 引擎推理服务\href{https://github.com/huawei-clouds/dls-mxserving-client/tree/master/java}{Java 客户端}。
\end{itemize}
\end{onehalfspacing}

\datedsubsection{\textbf{MOOC 用户行为分析}}{2016 年 10 月 – 2017 年 3 月}
\role{Python, Pandas, Scikit-learn, XGBoost} {毕业论文}
\begin{itemize}[parsep=0.5ex]
  \item 用户流失预测。基于用户点击浏览行为分析用户行为特征,对用户流失进行预测。
  \item 基于点击流的学习行为分析。基于用户点击内容和停留时间对点击流建模,挖掘和分析用户的行为模式。发表 \href{http://storage.cparty.co/user/02ct718cKf/activity/20161229092844J3VFZT9Q/publishnews/attachment_20170705055506001499234106971251_file}{一作EI 论文一篇}. Modeling and Interpreting User Navigation Patterns in MOOCs.
\end{itemize}


\datedsubsection{\textbf{360 新闻部实习}}{2016 年 3 月 – 2016 年 6 月}
\role{Python, Scala, Spark, Redis, MySQL }{新闻推荐算法工程师}
\begin{itemize}[parsep=0.5ex]
  \item 新闻 App 埋点日志校验,线上待推荐新闻类别和质量多维度分析统计。
  \item 推荐效果分析。统计分析每日新闻 CTR,Spark Job 批量生成用户画像,分析用户 Case。
  \item 用户画像增量更新。用户浏览日志离线分析,用户画像更新时间缩短 80\%。
\end{itemize}

% Reference Test
%\datedsubsection{\textbf{Paper Title\cite{zaharia2012resilient}}}{May. 2015}
%An xxx optimized for xxx\cite{verma2015large}
%\begin{itemize}
%  \item main contribution
%\end{itemize}

\section{\faCogs\ IT 技能}
% increase linespacing [parsep=0.5ex]
\begin{itemize}[parsep=0.5ex]
  \item 英语: 熟练 (六级 561)
  \item 开发: 熟悉 Linux 平台,编程语言 Java > C++ = Scala = Python,了解高可用服务系统架构,了解常见的机器学习算法,了解分布式系统原理,在校期间有Web开发和机器学习项目经验(省略)。
  \item GitHub: https://github.com/neuzxy
\end{itemize}

%\section{\faHeartO\ 获奖情况}
%\datedline{\textit{第一名}, xxx 比赛}{2013 年6 月}
%\datedline{其他奖项}{2015}
%
%\section{\faInfo\ 其他}
%% increase linespacing [parsep=0.5ex]
%\begin{itemize}[parsep=0.5ex]
%  \item 技术博客: http://blog.yours.me
%  \item GitHub: https://github.com/username
%  \item 语言: 英语 - 熟练(TOEFL xxx)
%\end{itemize}

%% Reference
%\newpage
%\bibliographystyle{IEEETran}
%\bibliography{mycite}
\end{document}
